\documentclass[11pt, a4paper]{article} % Dokumentenklasse: Artikel, 11pt Schriftgröße, A4-Papier

%-------------------------------------------------------------------------------
% GRUNDEINSTELLUNGEN: Seitenlayout, Sprache, Kodierung, Schriftart
%-------------------------------------------------------------------------------
\usepackage[a4paper,
            left=22mm, right=22mm,
            top=0.3cm, bottom=1cm,
            includehead, includefoot,
            headheight=3cm
            ]{geometry} % Zur Einstellung der Seitengeometrie
\usepackage[ngerman]{babel} % Deutsche Sprachanpassungen
\usepackage[T1]{fontenc}    % Wählt die T1-Schriftkodierung
\usepackage[utf8]{inputenc} % Erlaubt die direkte Eingabe von UTF-8 Zeichen
\usepackage{lmodern}        % Lädt die Latin Modern Schriftart
\usepackage{textcomp}       % Stellt zusätzliche Textsymbole bereit
\usepackage{gensymb}        % Stellt generische Symbole bereit

%-------------------------------------------------------------------------------
% MATHEMATIK
%-------------------------------------------------------------------------------
\usepackage{amsmath}        % Umfassende Erweiterungen für mathematische Formeln
\usepackage{amssymb}        % Zusätzliche mathematische Symbole
\usepackage{amsthm}         % Für Theorem-Umgebungen
\usepackage{latexsym}       % (Oft von amssymb abgedeckt)
\usepackage{derivative}     % Zum einfachen Setzen von Ableitungen
% \usepackage{verbatim}

%-------------------------------------------------------------------------------
% GRAFIKEN & DIAGRAMME
%-------------------------------------------------------------------------------
\usepackage{graphicx}       % Zum Einbinden von externen Grafiken
\usepackage{tikz}           % Mächtiges Paket zum Erstellen von Vektorgrafiken
\usepackage{circuitikz}     % Erweiterung für TikZ zum Zeichnen von Schaltplänen
\usepackage{subcaption}     % Ermöglicht Unterabbildungen und Untertabellen
\usepackage{caption}        % Zum Anpassen von Abbildungs- und Tabellenbeschriftungen
\usepackage{adjustbox}      % Zum Anpassen von Boxen
\usepackage{eso-pic}        % Ermöglicht das Platzieren von Elementen im Hintergrund/Vordergrund

%-------------------------------------------------------------------------------
% TABELLEN
%-------------------------------------------------------------------------------
\usepackage{booktabs}       % Erzeugt optisch ansprechendere Tabellenlinien
\usepackage{multirow}       % Ermöglicht das Verbinden von Zellen über mehrere Zeilen
\usepackage{colortbl}       % Ermöglicht das Einfärben von Tabellenzellen

%-------------------------------------------------------------------------------
% LISTEN & AUFZÄHLUNGEN
%-------------------------------------------------------------------------------
\usepackage{enumitem}       % Erweiterte Anpassungsmöglichkeiten für Listen

%-------------------------------------------------------------------------------
% KOPF-/FUSSZEILE & ABSCHNITTSFORMATIERUNG
%-------------------------------------------------------------------------------
\usepackage{fancyhdr}       % Zur Gestaltung von benutzerdefinierten Kopf- und Fußzeilen
\usepackage{lastpage}       % Stellt die Gesamtseitenzahl des Dokuments zur Verfügung
\usepackage{sectsty}        % Einfache Möglichkeit zur Änderung der Schriftart von Abschnittsüberschriften
\usepackage{titlesec}       % Erweiterte Kontrolle über das Aussehen von Abschnittsüberschriften
\usepackage{titletoc}       % Erweiterte Kontrolle über das Aussehen des Inhaltsverzeichnisses
\usepackage{lastpage}

%-------------------------------------------------------------------------------
% VERZEICHNISSE, REFERENZEN & LINKS
%-------------------------------------------------------------------------------
\usepackage{makeidx}        % Zur Erstellung eines Stichwortverzeichnisses (Index)
                            % -> Nicht vergessen: makeindex nach LaTeX-Lauf ausführen!
\makeindex                  % Index-Generierung aktivieren
\usepackage[colorlinks=true,
            linkcolor=blue,
            citecolor=green,
            urlcolor=magenta
            ]{hyperref}     % Erzeugt klickbare Links im PDF
                            % -> Sollte i.d.R. als eines der letzten Pakete geladen werden
\usepackage{chngcntr}       % Zum Ändern der Zähler-Zurücksetzung

%-------------------------------------------------------------------------------
% SPEZIELLE FORMATIERUNGEN & HILFSPAKETE
%-------------------------------------------------------------------------------
\usepackage[normalem]{ulem} % Bietet u.a. durchgestrichenen und unterstrichenen Text
\usepackage{calc}           % Ermöglicht Berechnungen innerhalb von Längenangaben
\usepackage{pdfpages}       % Zum Einbinden ganzer Seiten aus externen PDF-Dateien
\usepackage{blindtext}      % Zum Erzeugen von Blindtext und Beispiel-Dokumentstrukturen
\usepackage{float}          % Verbesserte Kontrolle über Gleitumgebungen
\usepackage{newfloat}       % Zum Definieren neuer Gleitumgebungstypen

%-------------------------------------------------------------------------------
% BENUTZERDEFINIERTE EINSTELLUNGEN & DEFINITIONEN
%-------------------------------------------------------------------------------

% Keine Einrückung am Absatzanfang
\setlength{\parindent}{0em}

% Setzt die Schriftart für alle Abschnittsüberschriften auf serifenlos
\allsectionsfont{\sffamily}

% Definition des Hintergrundbild-Befehls
\newcommand\BackgroundWave{%
    \put(0,0){%
        \parbox[b][\paperheight]{\paperwidth}{%
            \vfill %
            \centering %
            \vspace{12.0cm} %
            \includegraphics[width=\paperwidth]{logo/hsel-welle-grey}%
            \vfill %
        }%
    }%
}

\usepackage{xcolor}
\usepackage{mdframed}
% Definiere eine neue mdframed-Umgebung namens 'hinweisbox'
\newmdenv[
    linecolor=red,
    linewidth=1pt,
    frametitle=Hinweis,
    frametitlebackgroundcolor=red,
    frametitlefont=\color{white}\bfseries,
    frametitlerule=true,
    backgroundcolor=white,
    roundcorner=2pt,
]{hinweisbox}


% Definition neuer Gleitumgebungen
\DeclareFloatingEnvironment[
    listname  = {Diagrammverzeichnis},
    name      = Diagramm,
    fileext   = lod,
    placement = htp
]{diagram}

\DeclareFloatingEnvironment[
    listname  = {Oszillogrammverzeichnis},
    name      = Oszillogramm,
    fileext   = loo,
    placement = htp
]{oszillo}

%-------------------------------------------------------------------------------
% GLOBALE DEFINITIONEN
%-------------------------------------------------------------------------------
\newcommand{\praktikumstitel}{Praktikumstitel} %
\newcommand{\semester}{Semester X} %
\newcommand{\versuchsnummer}{\ } %
\newcommand{\versuchstitel}{Versuchstitel } %
\newcommand{\gruppe}{\ } %
\newcommand{\studiengang}{Studiengang Elektrotechnik}
\newcommand{\vornameStudEins}{Oliver}
\newcommand{\nachnameStudEins}{Schmidt}
\newcommand{\matrikelnummerStudEins}{7023462}
\newcommand{\vornameStudZwei}{\ } %
\newcommand{\nachnameStudZwei}{\ } %
\newcommand{\matrikelnummerStudZwei}{\ } %
\newcommand{\betreuerEins}{\ } %
\newcommand{\betreuerZwei}{\ } %

%-------------------------------------------------------------------------------
% DOKUMENTENBEGINN - Ab hier startet der eigentliche Inhalt
%-------------------------------------------------------------------------------
\begin{document}

% Beispielhafter Aufruf für das Hintergrundbild auf allen Seiten
% \AddToShipoutPicture*{\BackgroundWave}

\begin{titlepage}

	\hspace{-1.0cm}
	\begin{tabular}{p{8.0cm} p{8.0cm}}
		\includegraphics[width = 6.0cm]{logo/Technik.png} & % Logo
		\parbox[b]{8.0cm}{
		{\large  Fachbereich Technik }                      \\ % Fachbereich
		{\large  Abteilung Elektrotechnik und Informatik } % Abteilung
		}                                                   \\
		\\
		\hline
	\end{tabular}
	%
	\begin{center}

		\vspace{2.5cm}
		\LARGE{\textsc{\praktikumstitel\\ % Praktikumstitel (global definiert)
				\semester}}\\ % Semester (global definiert)

		\vspace{2cm}%
		\LARGE{\textsc{\versuchsnummer}}\\ % Versuchsnummer (global definiert)
		\LARGE{\versuchstitel} % Versuchstitel (global definiert)

		\vspace{4cm}%
		\large

		\gruppe\\ % Gruppe (global definiert)
		\studiengang\\\ \\ % Studiengang (global definiert)
		Vorgelegt von\\ % Einleitung Studierende

		\begin{table}[!ht]
			\centering
			\begin{tabular}{rll}
				\vornameStudEins & \nachnameStudEins & \matrikelnummerStudEins \\ % Studierender 1 (global definiert)
				\vornameStudZwei & \nachnameStudZwei & \matrikelnummerStudZwei % Studierender 2 (global definiert)
			\end{tabular}
		\end{table}


		\vspace{1cm}
		Emden, \today % Ort und Datum

		\vspace{1.5cm}%
		Betreut von\\ % Einleitung Betreuer
		\betreuerEins\\ % Betreuer 1 (global definiert)
		\betreuerZwei % Betreuer 2 (global definiert)

	\end{center}
	\normalsize
\end{titlepage}
    % Titelseite einbinden
\newpage
\pagestyle{fancy} % Aktiviert den benutzerdefinierten Seitenstil 'fancy'
\fancyhead[L]{{\includegraphics[width = 6.0cm]{logo/Technik.png}}} % Kopfzeile links: Logo einbinden
\fancyhead[R]{ % Kopfzeile rechts:
	\Large{{ % Große Schrift
				\textbf{\praktikumstitel}}}} % Fetter Praktikumstitel (global definiert)
\fancyfoot[L]{} % Fußzeile links: leer
\fancyfoot[C]{} % Fußzeile mitte: leer
\fancyfoot[R]{Seite \thepage} % Fußzeile rechts: Aktuelle Seitenzahl
\renewcommand{\footrulewidth}{0.4pt}% Dicke der Linie unter der Kopfzeile

\pagenumbering{Roman} % Seitennummerierung auf römische Ziffern setzen
\fancyfoot[L]{\leftmark} % Fußzeile links: Aktuelle Abschnittsüberschrift
\fancyfoot[C]{} % Fußzeile mitte: leer
\fancyfoot[R]{Seite \thepage\ von \pageref{LastPage}} % Fußzeile rechts: Aktuelle Seite von Gesamtseitenzahl
 % Kopf- und Fußzeilen-Definitionen einbinden
% Hier wäre ein guter Ort für \AddToShipoutPicture*{\BackgroundWave},
% wenn das Logo erst nach/mit der Kopf-/Fußzeile erscheinen soll.

\pagenumbering{Roman}        % Seitennummerierung auf Römisch umstellen
% Inhaltsverzeichnis
\iftoggle{showtoc}{
  \hypersetup{linkcolor=black}
  \tableofcontents
}{} % Das leere {} ist der else-Teil, falls du nichts anderes möchtest

% Zähler auf section-Ebene setzen (bleibt bestehen, da es die Nummerierung beeinflusst, nicht nur die Anzeige des Verzeichnisses)
\counterwithin{figure}{section}
\counterwithin{oszillo}{section}
\counterwithin{diagram}{section}
\counterwithin{table}{section}


% Tabellenverzeichnis
\iftoggle{showtables}{
  \hypersetup{linkcolor=black}
  \listoftables
  \addcontentsline{toc}{section}{Tabellenverzeichnis}
  % \newpage % Newpage nur wenn das Verzeichnis tatsächlich gezeigt wird und du es wünschst
}{}

% Abbildungsverzeichnis
\iftoggle{showfigures}{
  \listoffigures
  \addcontentsline{toc}{section}{Abbildungsverzeichnis}
  %\newpage
}{}

% Oszillogrammverzeichnis
\iftoggle{showoszillos}{
  \listofoszillo
  \addcontentsline{toc}{section}{Oszillogrammverzeichnis}
  %\newpage
}{}

% Diagrammverzeichnis
\iftoggle{showdiagrams}{
  \listofdiagram
  \addcontentsline{toc}{section}{Diagrammverzeichnis}
}{} % Das \newpage hier habe ich erstmal in den if-Block verschoben, falls Diagramme das letzte Verzeichnis sind.

% Das \hypersetup und \newpage am Ende ggf. anpassen oder in den letzten aktiven Block verschieben.
% Wenn das Diagrammverzeichnis das letzte ist UND angezeigt wird:
\iftoggle{showdiagrams}{
  \hypersetup{linkcolor=blue}
  \newpage
}{}

% Wenn du eine generelle Regel für das letzte angezeigte Verzeichnis brauchst,
% könnte man das \hypersetup und \newpage außerhalb der Bedingungen setzen
% oder in den letzten Block, der sicher angezeigt wird (z.B. nach dem Diagrammverzeichnis,
% aber dann wird es immer ausgeführt, es sei denn, man macht eine komplexere Bedingung).
% Für den Moment ist es so, dass die Linkfarbe nur dann auf Blau gesetzt und eine neue Seite begonnen wird,
% wenn das Diagrammverzeichnis angezeigt wird. Du kannst das anpassen.
% Eine einfachere Variante für das Zurücksetzen der Linkfarbe (wenn mind. ein Verzeichnis angezeigt wurde):
% \hypersetup{linkcolor=blue}
% Wenn du immer eine neue Seite nach den Verzeichnissen willst (falls welche da waren):
% \newpage % Dies würde aber auch eine neue Seite machen, wenn alle Verzeichnisse aus sind.
    % Verzeichnisse einbinden
% Stellt sicher, dass diese Datei \tableofcontents, \listoffigures,
% \listoftables, \listof{diagram}{...}, \listof{oszillo}{...} enthält
\newpage
%\ClearShipoutPicture

\pagenumbering{arabic}       % Seitennummerierung auf Arabisch umstellen

\section{Hier geht es los!}



% Evtl. Anhang, Literaturverzeichnis, Index hier einbinden
% \appendix
% \include{src/anhang}
% \bibliography{meine_bib_datei}
% \bibliographystyle{plain}
\printindex                  % Index ausgeben

\end{document}
